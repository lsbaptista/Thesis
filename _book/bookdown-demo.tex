% Options for packages loaded elsewhere
\PassOptionsToPackage{unicode}{hyperref}
\PassOptionsToPackage{hyphens}{url}
%
\documentclass[
  12pt,
  a4paper,
  openany]{book}
\usepackage{lmodern}
\usepackage{setspace}
\usepackage{amssymb,amsmath}
\usepackage{ifxetex,ifluatex}
\ifnum 0\ifxetex 1\fi\ifluatex 1\fi=0 % if pdftex
  \usepackage[T1]{fontenc}
  \usepackage[utf8]{inputenc}
  \usepackage{textcomp} % provide euro and other symbols
\else % if luatex or xetex
  \usepackage{unicode-math}
  \defaultfontfeatures{Scale=MatchLowercase}
  \defaultfontfeatures[\rmfamily]{Ligatures=TeX,Scale=1}
  \setmainfont[]{Arial}
\fi
% Use upquote if available, for straight quotes in verbatim environments
\IfFileExists{upquote.sty}{\usepackage{upquote}}{}
\IfFileExists{microtype.sty}{% use microtype if available
  \usepackage[]{microtype}
  \UseMicrotypeSet[protrusion]{basicmath} % disable protrusion for tt fonts
}{}
\usepackage{xcolor}
\IfFileExists{xurl.sty}{\usepackage{xurl}}{} % add URL line breaks if available
\IfFileExists{bookmark.sty}{\usepackage{bookmark}}{\usepackage{hyperref}}
\hypersetup{
  hidelinks,
  pdfcreator={LaTeX via pandoc}}
\urlstyle{same} % disable monospaced font for URLs
\usepackage[left=3.5cm,right=2cm,top=3cm,bottom=3cm, asymmetric]{geometry}
\usepackage{color}
\usepackage{fancyvrb}
\newcommand{\VerbBar}{|}
\newcommand{\VERB}{\Verb[commandchars=\\\{\}]}
\DefineVerbatimEnvironment{Highlighting}{Verbatim}{commandchars=\\\{\}}
% Add ',fontsize=\small' for more characters per line
\usepackage{framed}
\definecolor{shadecolor}{RGB}{248,248,248}
\newenvironment{Shaded}{\begin{snugshade}}{\end{snugshade}}
\newcommand{\AlertTok}[1]{\textcolor[rgb]{0.94,0.16,0.16}{#1}}
\newcommand{\AnnotationTok}[1]{\textcolor[rgb]{0.56,0.35,0.01}{\textbf{\textit{#1}}}}
\newcommand{\AttributeTok}[1]{\textcolor[rgb]{0.77,0.63,0.00}{#1}}
\newcommand{\BaseNTok}[1]{\textcolor[rgb]{0.00,0.00,0.81}{#1}}
\newcommand{\BuiltInTok}[1]{#1}
\newcommand{\CharTok}[1]{\textcolor[rgb]{0.31,0.60,0.02}{#1}}
\newcommand{\CommentTok}[1]{\textcolor[rgb]{0.56,0.35,0.01}{\textit{#1}}}
\newcommand{\CommentVarTok}[1]{\textcolor[rgb]{0.56,0.35,0.01}{\textbf{\textit{#1}}}}
\newcommand{\ConstantTok}[1]{\textcolor[rgb]{0.00,0.00,0.00}{#1}}
\newcommand{\ControlFlowTok}[1]{\textcolor[rgb]{0.13,0.29,0.53}{\textbf{#1}}}
\newcommand{\DataTypeTok}[1]{\textcolor[rgb]{0.13,0.29,0.53}{#1}}
\newcommand{\DecValTok}[1]{\textcolor[rgb]{0.00,0.00,0.81}{#1}}
\newcommand{\DocumentationTok}[1]{\textcolor[rgb]{0.56,0.35,0.01}{\textbf{\textit{#1}}}}
\newcommand{\ErrorTok}[1]{\textcolor[rgb]{0.64,0.00,0.00}{\textbf{#1}}}
\newcommand{\ExtensionTok}[1]{#1}
\newcommand{\FloatTok}[1]{\textcolor[rgb]{0.00,0.00,0.81}{#1}}
\newcommand{\FunctionTok}[1]{\textcolor[rgb]{0.00,0.00,0.00}{#1}}
\newcommand{\ImportTok}[1]{#1}
\newcommand{\InformationTok}[1]{\textcolor[rgb]{0.56,0.35,0.01}{\textbf{\textit{#1}}}}
\newcommand{\KeywordTok}[1]{\textcolor[rgb]{0.13,0.29,0.53}{\textbf{#1}}}
\newcommand{\NormalTok}[1]{#1}
\newcommand{\OperatorTok}[1]{\textcolor[rgb]{0.81,0.36,0.00}{\textbf{#1}}}
\newcommand{\OtherTok}[1]{\textcolor[rgb]{0.56,0.35,0.01}{#1}}
\newcommand{\PreprocessorTok}[1]{\textcolor[rgb]{0.56,0.35,0.01}{\textit{#1}}}
\newcommand{\RegionMarkerTok}[1]{#1}
\newcommand{\SpecialCharTok}[1]{\textcolor[rgb]{0.00,0.00,0.00}{#1}}
\newcommand{\SpecialStringTok}[1]{\textcolor[rgb]{0.31,0.60,0.02}{#1}}
\newcommand{\StringTok}[1]{\textcolor[rgb]{0.31,0.60,0.02}{#1}}
\newcommand{\VariableTok}[1]{\textcolor[rgb]{0.00,0.00,0.00}{#1}}
\newcommand{\VerbatimStringTok}[1]{\textcolor[rgb]{0.31,0.60,0.02}{#1}}
\newcommand{\WarningTok}[1]{\textcolor[rgb]{0.56,0.35,0.01}{\textbf{\textit{#1}}}}
\usepackage{longtable,booktabs}
% Correct order of tables after \paragraph or \subparagraph
\usepackage{etoolbox}
\makeatletter
\patchcmd\longtable{\par}{\if@noskipsec\mbox{}\fi\par}{}{}
\makeatother
% Allow footnotes in longtable head/foot
\IfFileExists{footnotehyper.sty}{\usepackage{footnotehyper}}{\usepackage{footnote}}
\makesavenoteenv{longtable}
\usepackage{graphicx}
\makeatletter
\def\maxwidth{\ifdim\Gin@nat@width>\linewidth\linewidth\else\Gin@nat@width\fi}
\def\maxheight{\ifdim\Gin@nat@height>\textheight\textheight\else\Gin@nat@height\fi}
\makeatother
% Scale images if necessary, so that they will not overflow the page
% margins by default, and it is still possible to overwrite the defaults
% using explicit options in \includegraphics[width, height, ...]{}
\setkeys{Gin}{width=\maxwidth,height=\maxheight,keepaspectratio}
% Set default figure placement to htbp
\makeatletter
\def\fps@figure{htbp}
\makeatother
\setlength{\emergencystretch}{3em} % prevent overfull lines
\providecommand{\tightlist}{%
  \setlength{\itemsep}{0pt}\setlength{\parskip}{0pt}}
\setcounter{secnumdepth}{5}
\pagestyle{plain}
\usepackage{booktabs}
\usepackage{amsmath,amsfonts,amsthm,bm} % Math packages
\usepackage{fontspec}
\usepackage[utf8]{inputenc}
\usepackage[portuguese]{babel}

\usepackage{etoolbox}

\usepackage{titlesec}

\usepackage{makeidx}
\makeindex
\usepackage[nottoc]{tocbibind}

\makeatletter
\patchcmd{\ttl@save@mkschap}{*}{}{}{}
\makeatother
\usepackage{indentfirst}
\usepackage{booktabs}
\usepackage{longtable}
\usepackage{array}
\usepackage{multirow}
\usepackage{wrapfig}
\usepackage{float}
\usepackage{colortbl}
\usepackage{pdflscape}
\usepackage{tabu}
\usepackage{threeparttable}
\usepackage{threeparttablex}
\usepackage[normalem]{ulem}
\usepackage{makecell}
\ifluatex
  \usepackage{selnolig}  % disable illegal ligatures
\fi
\usepackage[]{natbib}
\bibliographystyle{apalike}

\author{}
\date{\vspace{-2.5em}}

\begin{document}

%titlepage
\thispagestyle{empty}
\begin{center}
\begin{minipage}{0.90\linewidth}
    \centering
%University logo
    \includegraphics[width=1\textwidth]{image/biglogo.png}

%Thesis title
    {\uppercase{\Large Impacto da volatilidade na otimização de portfolios financeiros\par}}
    \vspace{2cm}
%Author's name
    {\Large Leonel da Silva Baptista\par}
    \vspace{2cm}
%Degree
    {\Large Mestrado em Estatística, Matemática e Computação\par}
    {\Large Ramo Estatística Computacional\par}
    \vspace{2cm}

%Date
    {\Large 2020}
\end{minipage}
\end{center}


%titlepage
\thispagestyle{empty}
\begin{center}
\begin{minipage}{0.90\linewidth}
    \centering
%University logo
    \includegraphics[width=1\textwidth]{image/logo.png}

%Thesis title
    {\uppercase{\Large Impacto da volatilidade na otimização de portfolios financeiros\par}}
    \vspace{2cm}
%Author's name
    {\Large Leonel da Silva Baptista\par}
    \vspace{2cm}
%Degree
    {\Large Mestrado em Estatística, Matemática e Computação\par}
    {\Large Ramo Estatística Computacional\par}
    \vspace{2cm}
%oriented
    {\Large Dissertação orientada pelo\par}
    {\Large Professor Doutor Amílcar Manuel do Rosário Oliveira\par}
    \vspace{2cm}
%Date
    {\Large 2020}
\end{minipage}
\end{center}

\pagenumbering{gobble}% Remove page numbers (and reset to 1)

\clearpage

\chapter*{Resumo}
\fontsize{12}{21}\selectfont
A presente dissertação têm como âmbito a análise de três métodos diferentes de obter a volatilidade de instrumentos financeiros, nomeadamente valores mobiliários, e seu consequente impacto no resultado da rentabilidade de portfolios constituídos utilizando como pressuposto a média variância, assim como a sua exposição ao risco, sendo que a volatilidade constitui uma peça central na constituição de determinados instrumentos financeiros e respetivo calculo de exposição ao risco.

Os métodos analisados para calculo da volatilidade são a Média Movel Exponencial (EWMA), o modelo da heteroscedasticidade condicional auto-regressiva generalizada (GARCH) e a volatilidade implícita. Os dois primeiros métodos têm como subjacentes dados históricos dos instrumentos financeiros, sendo que a volatilidade implícita é a volatilidade esperada pelo mercado, sendo obtida através da cotação das opções dos respetivos subjacentes.

A análise de risco é efetuada aplicando dois métodos complementares de análise. O Value at Risk (VaR), que contempla a percentagem de perdas que excedem o VaR e o Expected shortfall (ES), que contempla a magnitude dessas perdas.

Este trabalho é realizado tendo como ferramenta de apoio a linguagem de programação R.
\bigbreak

\noindent\textbf{Palavras chave:} Volatilidade, portfolio, rentabilidade, risco, R




\pagenumbering{roman}% Arabic page numbers (and reset to 1)
\setcounter{page}{2}

\chapter*{Abstract}
\fontsize{12}{21}\selectfont
The scope of this dissertation is to analyze three different methods of obtaining the volatility of financial instruments, namely securities, and their consequent impact on the profitability of portfolios constituted using the assumption of mean-variance, as well as their exposure to risk, volatility is a central element in the constitution of certain financial instruments and the respective calculation of exposure to risk.

The methods analyzed for calculating the volatility are the Exponential Moving Average (EWMA), the generalized autoregressive conditional heteroscedasticity model (GARCH) and the implied volatility. The first two methods are based on historical data on financial instruments, the implied volatility being the volatility expected by the market, being obtained through the quotation of the options of the respective underlying.

The risk analysis is carried out using two complementary methods of analysis. Value at Risk (VaR), which includes the percentage of losses that exceed VaR and Expected shortfall (ES), which considers the magnitude of these losses.

This work is carried out using the R programming language as a support tool.

\bigbreak

\noindent\textbf{Keywords:} Volatility, portfolio, profitability, risk, R

\newenvironment{dedication}
  {\clearpage           % we want a new page
   \itshape             % the text is in italics
   \raggedleft          % flush to the right margin
  }
  {\par % end the paragraph
   \vspace{\stretch{3}} % space at bottom is three times that at the top
   \clearpage           % finish off the page
  }
\begin{dedication}
{\Large Dedicado a minha esposa\par}
\end{dedication}

\chapter*{Agradecimentos}

\renewcommand*\contentsname{Índice}
{
\setcounter{tocdepth}{2}
\tableofcontents
}
\listoftables
\listoffigures
\setstretch{1.5}
\hypertarget{simbologia-e-notauxe7uxf5es}{%
\chapter*{Simbologia e notações}\label{simbologia-e-notauxe7uxf5es}}

\mainmatter

\hypertarget{intro}{%
\chapter*{Introdução}\label{intro}}
\addcontentsline{toc}{chapter}{Introdução}

A estatística aplicada ao sector financeiro têm sido pratica comum nas últimas décadas, sendo que a sua aplicação não se resume apenas a estatística descritiva, tendo vindo a beneficiar dos avanços verificados na aplicação de ferramentas estatísticas para analise preditiva dos dados, devido essencialmente aos avanços tecnológicos no hardware de equipamentos informáticos que permitem a aplicação de algoritmos mais complexos que de outro modo não seria possível utilizar, pelo menos em tempo útil.

Um dos sectores com grande aplicabilidade dos métodos matemáticos e estatísticos nas finanças é a analise quantitativa, sendo as principais áreas de aplicação a estruturação de derivados, gestão do risco, trading automático e gestão de investimentos.

Historicamente, a analise quantitativa iniciou-se em 1900 com Louis Jean-Baptiste Alphonse Bachelier, onde a sua tese de doutoramento forneceu um modelo para estipular o preço de opções considerando uma distribuição normal (\url{https://en.wikipedia.org/wiki/Quantitative_analysis_(finance)}).

Na década de 50, Harry Markowitz escreveu um artigo intitulado ``Portfolio Selection'' que viria a revolucionar o modo como selecionar uma carteira de instrumentos financeiros, aplicando princípios de correlação e variância de modo a constituir portfolios de ações , onde a ``fronteira eficiente'' representa portfolios que maximizam retornos de acordo com o risco assumido, providenciando modelos que demonstravam que só com a diversificação de investimentos é que se conseguiria atingir a eficiência, embora só bastante mais tarde esta teoria começa-se a ver a sua aplicabilidade nas instituições financeiras. A aplicabilidade de técnicas estatísticas fica saliente quando Markowitz estipula que ``para usar a regra da média-variância na seleção de ações devemos ter procedimentos para encontrar \(\mu_i\) e \(\sigma_{ij}\) razoáveis. Estes procedimentos, eu acredito, devem combinar técnicas estatísticas e julgamento prático do Homem'' \citep[pp.91]{Markowitz1952}. As limitações do modelo prendem-se com pressuposto que não representam exatamente a realidade, como o pressuposto de que os retornos das ações seguem uma distribuição normal, sendo que a distribuição dos retornos segue muitas vezes uma distribuição com curtose leptocúrtica, apresentando caudas pesadas e um pico superior ao da distribuição normal.

Na década de 60, Sharpe, Lintner and Mossin desenvolveram um modelo para equilíbrio de mercado, definido como \emph{Capital Asset Pricing Model} (CAPM), descrevendo a relação entre risco sistemático e o retorno esperado. O modelo pressupõe que, se todos os investidores contêm o mesmo portfolio, então, em equilíbrio esse deve ser o portfolio de mercado. De acordo com \citet{Sharpe1964}, em equilíbrio os preços dos ativos de capital foram ajustados de forma que o investidor consiga atingir qualquer ponto desejado ao longo da reta do mercado capital, ou \emph{capital market line} (CMP), pressupondo que o investidor siga uma estratégia de diversificação do investimento. O CMP pode ser utilizado de modo a otimizar um portfolio caso seja contemplado uma taxa de juro sem risco na sua estruturação, sendo o ponto tangente a curva denominada de fronteira eficiente.

Também na década de 60 foi apresentado pela primeira vez o modelo de Black-Scholes-Merton, fornecendo uma solução para valorizar opções europeias e outros derivados. O modelo assume que os preços têm uma distribuição lognormal e que a volatilidade é constante ao longo do tempo. A volatilidade que é assumida neste modelo é a volatilidade implícita da opção, ou seja, a volatilidade para o qual o valor dado pelo modelo Black-Scholes-Merton iguala o preço de mercado. Outros modelos foram desenvolvidos de modo a valorizar opções e outros derivados financeiros, entre eles a arvore binomial e simulação de Monte Carlo.

Os vários modelos apresentados têm como finalidade contribuir para uma decisão mais informada por parte do investidor, sendo que na generalidade o investidor irá optar pelo investimento que apresenta um maior retorno de acordo com o risco a que se dispõem estar exposto, tendo em consideração o seu perfil e expectativas.A escolha do portfólio é feita resolvendo um problema de otimização, no qual o risco do portfólio é minimizado sendo definido como retrição o valor desejado de retorno esperado. Desta forma é importante quantificar o risco. Existem vários modelos para quantificar o risco, ou seja quantificar a perda esperada de acordo com a hipótese de ocorrência de determinado cenário, sendo que 2 desses modelos para análise do risco são o \emph{Value at Risk} (VaR) e o \emph{Expected Shortfall}, também denominado \emph{Conditional Value at Risk} (CVaR). De acordo com \citet{HistVaR}, em 1922 no New York Stock Exchange já eram exigidos requisitos de capital a alguns dos seus membros, tendo na década de 50 Markowitz e Roy, separadamente, publicado metodos para quantificar o VaR, sendo ambos bastante similares e com a finalidade de quantificar o risco a que estaria exposto um portfolio. A necessidade de utilizar medidas de risco mais sofisticada tornou-se mais visível na década de 80 devido ao aparecimento de produtos mais complexos e ao aumento da volatilidade dos mercados, sendo que devido a regulamentação cada vez mais exigente, como o Basel III e Solvency II, as instituições financeiras e seguradoras devem implementar mecanismos de gestão de risco, sendo que os requisitos de capital são bastante mais exigentes desde a ocorrência da crise do subprime em 2007. De acordo com os acordos de Basel o VaR deve ser estimado diariamente utilizando o percentil 99\textsuperscript{th}.

Todos estes modelos têm tido aplicabilidade na análise quantitativa financeira, sendo que novos modelos foram sendo desenvolvidos ou apenas melhorados de modo a dar resposta à realidade verificada no mercado. Como podemos subentender uma das disciplinas fundamentais na definição destes modelos é o conhecimento de estatística e a sua aplicação prática, quer através de técnicas paramétricas, onde se assume um pressuposto forte de que os valores de uma variável têm uma distribuição normal, seja através de técnicas não paramétricas, onde não se assume que a distribuição dos valores de uma variável apresentam distribuição normal.

O trabalho desenvolvido ao longo desta tese, propõem-se analisar o impacto que poderá ter o método de calculo da volatilidade sobre a definição de um portfolio, o seu retorno esperado assim como quantificar o risco a que se estará exposto. Desde os primeiros trabalhos de \citet{Markowitz1952} acerca da otimização de portfolios que vários outros trabalhos foram desenvolvidos para constituição e otimização de portfolios, sendo que nesta dissertação se irá aplicar o método introduzido por \citet{Markowitz1952}, utilizando a teoria da média-variância para a constituição de um portfolio. O mesmo se poderá afirmar acerca do cálculo da exposição ao risco por um portfolio, sendo, no entanto, o VaR e o CVaR dois dos métodos mais utilizados para quantificar essa métrica, sendo que de acordo com \citet{OptVaR2000} o CVaR é conhecido por ter melhores propriedades que o VaR. Como iremos ver existem diferentes métodos de calculo do VaR e CVaR, sendo que neste trabalho iremos utilizar a que for mais adequada ao tipo de dados em análise.

Como inicialmente referido, o objetivo é analisar as diferentes formas de calcular a variância, sendo que iremos focar-nos em três formas diferentes de calcular esse valor, análisar a sua implicação no resultado final, assim como o valor que resulta da quantificação de risco e o seu impacto na perceção pelo investidor.

A investigação decorrerá aplicando-se os diferentes métodos de calculo da variância a dados do mercado, assim como os pressupostos teóricos de cada um dos métodos, sendo constituída uma carteira de ações integrantes do Euro Stoxx 50, calculando o seu retorno final para cada uma das volatilidades, assim como o risco a que está exposto um investidor. O retorno obtido será também comparado com um \emph{benchmark}, neste caso o Euro Stoxx 50.

Ao longo deste percurso será também análisada a forma como é constituídos um portfolio, a teoria subjacente a média-variância, as retrições aplicáveis ao modelo desenvolvido, assim como a teoria subjacente aos vários métodos para análise da exposição ao risco aquando da constituição de um portfolio.

A aplicação pratica dos métodos aos dados do mercado será realizado com o apoio da linguagem de programação R, utilizando para esse efeito os vários ``packages'' disponíveis para aplicação ao sector financeiro. Um dos capítulos será dedicado a apresentação e descrição dos principais ``packages'' utilizados para análise dos dados, sendo que o R é parte integrante desta dissertação como ferramenta de análise estatísticas e aplicabilidade ao sector financeiro.

\begingroup
\titleformat{\chapter}[display]
{\normalfont\huge\bfseries\centering}{\chaptertitlename\ \thechapter}{20pt}{\Huge}

\hypertarget{modelauxe7uxe3o-estatuxedstica-na-otimizauxe7uxe3o-de-portfuxf3lios}{%
\chapter{Modelação Estatística na Otimização de Portfólios}\label{modelauxe7uxe3o-estatuxedstica-na-otimizauxe7uxe3o-de-portfuxf3lios}}

\newpage

\hypertarget{series-temporais}{%
\section{Series Temporais}\label{series-temporais}}

\hypertarget{portfolios-muxe9dia-variuxe2ncia}{%
\section{Portfolios média-variância}\label{portfolios-muxe9dia-variuxe2ncia}}

A otimização de portfolios através da diversificação é um conceito básico que teve origem em Markowitz, criando o conceito de fronteira eficiente. Existem vários pressupostos definidos na obtenção deste modelo, não sendo, no entanto, o âmbito aqui analisar esses mesmos pressupostos, considerando que independentemente disso esses mesmos pressupostos são verificados. De acordo com \citet{Modern2013}, ``todas os pressupostos acerca da analise de portfolio foram demonstradas serem simplistas, e em alguns casos demasiado simplistas.{[}\ldots{]}Pessoas necessitam apenas de se comportar como se fossem descritas pelos pressupostos para uma teoria ser válida'' \citep[pp.5]{Modern2013}.

Para aplicação deste modelo deve-se obter os seguintes dados dos instrumentos financeiros que vão constituir o portfolio:

\begin{itemize}
\tightlist
\item
  A taxa do retorno esperado, E(r);
\item
  O desvio padrão dos retornos, \(\sigma\);
\item
  O coeficiente de correlação,\(\rho\), entre cada um dos activos.
\end{itemize}

As séries dos retornos diários para cada um dos activos é calculada de acordo com a seguinte formula:

\begin{equation} 
  R_i = ln\Big(\frac{V_f}{V_i}\Big)
  \label{eq:log}
\end{equation}

A média dos retornos de cada um dos activos é calculada de acordo com a sua média aritmética:

\begin{equation} 
  \overline{R} = \frac{\displaystyle\sum_{i=1}^n R_i}{n}
  \label{eq:mean}
\end{equation}

A variância de cada um dos activos é calculada usando 3 métodos diferentes que passamos seguidamente a descrever:

\begin{itemize}
\tightlist
\item
  \textbf{Média Móvel Exponencial (EWMA)}
\end{itemize}

A média móvel exponencial, referida a partir de agora como EWMA, é um método de calculo da variância, sendo um método que utiliza os dados históricos associando mais peso aos retornos mais recentes no calculo da variância.

\begin{itemize}
\tightlist
\item
  \textbf{Heteroscedasticidade condicional auto-regressiva generalizada (GARCH)}
\end{itemize}

\hypertarget{pacotes-do-r-para-anuxe1lise}{%
\chapter{Pacotes do R para análise}\label{pacotes-do-r-para-anuxe1lise}}

\newpage

We describe our methods in this chapter.

\hypertarget{aplicauxe7uxe3o-a-dados-do-modelo}{%
\chapter{Aplicação a dados do modelo}\label{aplicauxe7uxe3o-a-dados-do-modelo}}

\endgroup
\newpage

\hypertarget{aplicauxe7uxe3o-a-dados-do-modelo-1}{%
\section{Aplicação a dados do modelo}\label{aplicauxe7uxe3o-a-dados-do-modelo-1}}

A primeira fase consiste em extrair de forma aleatória 14 acções constituintes do Euro Stoxx 50, estando os valores obtidos representados na tabela \ref{tab:nice-tab}.
\scriptsize

\begin{Shaded}
\begin{Highlighting}[]
\NormalTok{mydata \textless{}{-}}\StringTok{ }\KeywordTok{read.xlsx}\NormalTok{(}\StringTok{"data/IndexConstituentsstoxx\_50.xlsx"}\NormalTok{, }\DecValTok{1}\NormalTok{)}
\KeywordTok{set.seed}\NormalTok{(}\DecValTok{14}\NormalTok{)}
\NormalTok{dt \textless{}{-}}\StringTok{ }\KeywordTok{sample}\NormalTok{(mydata}\OperatorTok{$}\NormalTok{Full.name, }\DataTypeTok{size =} \DecValTok{14}\NormalTok{,}\DataTypeTok{replace =}\NormalTok{ F)}
\NormalTok{d \textless{}{-}}\StringTok{ }\NormalTok{knitr}\OperatorTok{::}\KeywordTok{kable}\NormalTok{(}
\NormalTok{  dt, }\DataTypeTok{caption =} \StringTok{"Empresas extraidas do Euro Stoxx 50"}\NormalTok{,}
  \DataTypeTok{booktabs =} \OtherTok{TRUE}
\NormalTok{)}
\KeywordTok{kable\_styling}\NormalTok{(d, }\DataTypeTok{latex\_options =} \StringTok{"hold\_position"}\NormalTok{, }\DataTypeTok{position =} \StringTok{"center"}\NormalTok{)}
\end{Highlighting}
\end{Shaded}

\begin{table}[!h]

\caption{\label{tab:nice-tab}Empresas extraidas do Euro Stoxx 50}
\centering
\begin{tabular}[t]{l}
\toprule
x\\
\midrule
UNILEVER ORD\\
SOCIETE GENERALE ORD\\
BAYER N ORD\\
TELEFONICA ORD\\
ENEL ORD\\
\addlinespace
DEUTSCHE TELEKOM N ORD\\
AB INBEV ORD\\
DAIMLER N ORD\\
ORANGE ORD\\
SANOFI ORD\\
\addlinespace
AIRBUS ORD\\
INTESA SANPAOLO ORD\\
AHOLD DEL ORD\\
ASML HOLDING ORD\\
\bottomrule
\end{tabular}
\end{table}

\normalsize

Reference a figure by its code chunk label with the \texttt{fig:} prefix, e.g., see Figure \ref{fig:nice-fig}.
\scriptsize

\begin{figure}
\includegraphics[width=0.3\linewidth]{bookdown-demo_files/figure-latex/nice-fig-1} \includegraphics[width=0.3\linewidth]{bookdown-demo_files/figure-latex/nice-fig-2} \includegraphics[width=0.3\linewidth]{bookdown-demo_files/figure-latex/nice-fig-3} \includegraphics[width=0.3\linewidth]{bookdown-demo_files/figure-latex/nice-fig-4} \includegraphics[width=0.3\linewidth]{bookdown-demo_files/figure-latex/nice-fig-5} \includegraphics[width=0.3\linewidth]{bookdown-demo_files/figure-latex/nice-fig-6} \includegraphics[width=0.3\linewidth]{bookdown-demo_files/figure-latex/nice-fig-7} \includegraphics[width=0.3\linewidth]{bookdown-demo_files/figure-latex/nice-fig-8} \includegraphics[width=0.3\linewidth]{bookdown-demo_files/figure-latex/nice-fig-9} \includegraphics[width=0.3\linewidth]{bookdown-demo_files/figure-latex/nice-fig-10} \includegraphics[width=0.3\linewidth]{bookdown-demo_files/figure-latex/nice-fig-11} \includegraphics[width=0.3\linewidth]{bookdown-demo_files/figure-latex/nice-fig-12} \includegraphics[width=0.3\linewidth]{bookdown-demo_files/figure-latex/nice-fig-13} \includegraphics[width=0.3\linewidth]{bookdown-demo_files/figure-latex/nice-fig-14} \caption{Here is a nice figure!}\label{fig:nice-fig}
\end{figure}

Calculo do retorno diario logaritmico
\scriptsize

\normalsize

Correlação entre activos
\scriptsize

\begin{Shaded}
\begin{Highlighting}[]
\CommentTok{\#do not print hash}
\KeywordTok{options}\NormalTok{(}\DataTypeTok{width =} \DecValTok{70}\NormalTok{)}
\KeywordTok{cor}\NormalTok{(returns)}
\end{Highlighting}
\end{Shaded}

\begin{verbatim}
                 TELEFONICA  UNILEVER    SOCGEN     BAYER      ENEL
TELEFONICA        1.0000000 0.4424126 0.6117442 0.6410991 0.5038660
UNILEVER          0.4424126 1.0000000 0.2973260 0.4372588 0.4401667
SOCGEN            0.6117442 0.2973260 1.0000000 0.5607855 0.5056762
BAYER             0.6410991 0.4372588 0.5607855 1.0000000 0.4962673
ENEL              0.5038660 0.4401667 0.5056762 0.4962673 1.0000000
DEUTSCHE TELEKOM  0.6164856 0.4861206 0.4951897 0.5892044 0.5751642
AB INBEV          0.4567942 0.4748401 0.4887377 0.4612602 0.5266856
DAIMLER           0.5714944 0.3770780 0.6711539 0.5390019 0.6159233
ORANGE            0.6711857 0.4175865 0.4873755 0.6019032 0.4907272
SANOFI            0.4859927 0.4720770 0.4047847 0.5037108 0.5685078
AIRBUS            0.3725871 0.3383565 0.5167281 0.3721095 0.4449917
INTESA            0.6296856 0.3094075 0.7600133 0.6756706 0.4945083
AHOLD             0.3623656 0.3639984 0.2836877 0.3756120 0.3664813
ASML              0.3619936 0.3287217 0.3843967 0.3624049 0.4757841
                 DEUTSCHE TELEKOM  AB INBEV   DAIMLER    ORANGE
TELEFONICA              0.6164856 0.4567942 0.5714944 0.6711857
UNILEVER                0.4861206 0.4748401 0.3770780 0.4175865
SOCGEN                  0.4951897 0.4887377 0.6711539 0.4873755
BAYER                   0.5892044 0.4612602 0.5390019 0.6019032
ENEL                    0.5751642 0.5266856 0.6159233 0.4907272
DEUTSCHE TELEKOM        1.0000000 0.5150677 0.5688911 0.6770446
AB INBEV                0.5150677 1.0000000 0.5411385 0.4447014
DAIMLER                 0.5688911 0.5411385 1.0000000 0.4968570
ORANGE                  0.6770446 0.4447014 0.4968570 1.0000000
SANOFI                  0.5048925 0.4266473 0.4429070 0.4656332
AIRBUS                  0.4386247 0.5037180 0.5308221 0.3383876
INTESA                  0.5049104 0.4287855 0.6008106 0.5260441
AHOLD                   0.4161850 0.3105371 0.3300496 0.3907452
ASML                    0.4245982 0.4294115 0.4963008 0.3513143
                    SANOFI    AIRBUS    INTESA     AHOLD      ASML
TELEFONICA       0.4859927 0.3725871 0.6296856 0.3623656 0.3619936
UNILEVER         0.4720770 0.3383565 0.3094075 0.3639984 0.3287217
SOCGEN           0.4047847 0.5167281 0.7600133 0.2836877 0.3843967
BAYER            0.5037108 0.3721095 0.6756706 0.3756120 0.3624049
ENEL             0.5685078 0.4449917 0.4945083 0.3664813 0.4757841
DEUTSCHE TELEKOM 0.5048925 0.4386247 0.5049104 0.4161850 0.4245982
AB INBEV         0.4266473 0.5037180 0.4287855 0.3105371 0.4294115
DAIMLER          0.4429070 0.5308221 0.6008106 0.3300496 0.4963008
ORANGE           0.4656332 0.3383876 0.5260441 0.3907452 0.3513143
SANOFI           1.0000000 0.3564244 0.4111681 0.3871515 0.4125347
AIRBUS           0.3564244 1.0000000 0.4317871 0.2223268 0.4525017
INTESA           0.4111681 0.4317871 1.0000000 0.3028476 0.3510794
AHOLD            0.3871515 0.2223268 0.3028476 1.0000000 0.2605924
ASML             0.4125347 0.4525017 0.3510794 0.2605924 1.0000000
\end{verbatim}

\normalsize

\scriptsize

\begin{Shaded}
\begin{Highlighting}[]
\KeywordTok{par}\NormalTok{(}\DataTypeTok{mar =} \KeywordTok{c}\NormalTok{(}\DecValTok{1}\NormalTok{,}\DecValTok{1}\NormalTok{,}\DecValTok{1}\NormalTok{,}\DecValTok{1}\NormalTok{))}
\KeywordTok{plot}\NormalTok{(returns}\OperatorTok{$}\NormalTok{TELEFONICA, }\DataTypeTok{main =} \StringTok{"TEF.MC"}\NormalTok{,}\DataTypeTok{col=}\StringTok{"red"}\NormalTok{)}
\KeywordTok{plot}\NormalTok{(returns}\OperatorTok{$}\NormalTok{UNILEVER, }\DataTypeTok{main =} \StringTok{"UNA.AS"}\NormalTok{,}\DataTypeTok{col=}\StringTok{"red"}\NormalTok{)}
\KeywordTok{plot}\NormalTok{(returns}\OperatorTok{$}\NormalTok{SOCGEN, }\DataTypeTok{main =} \StringTok{"GLE.PA"}\NormalTok{,}\DataTypeTok{col=}\StringTok{"red"}\NormalTok{)}
\KeywordTok{plot}\NormalTok{(returns}\OperatorTok{$}\NormalTok{BAYER, }\DataTypeTok{main =} \StringTok{"ENEL.MI"}\NormalTok{,}\DataTypeTok{col=}\StringTok{"red"}\NormalTok{)}
\KeywordTok{plot}\NormalTok{(returns}\OperatorTok{$}\NormalTok{ENEL, }\DataTypeTok{main =} \StringTok{"BAYN.DE"}\NormalTok{,}\DataTypeTok{col=}\StringTok{"red"}\NormalTok{)}
\KeywordTok{plot}\NormalTok{(returns}\OperatorTok{$}\StringTok{\textasciigrave{}}\DataTypeTok{DEUTSCHE TELEKOM}\StringTok{\textasciigrave{}}\NormalTok{, }\DataTypeTok{main =} \StringTok{"DTE.DE"}\NormalTok{,}\DataTypeTok{col=}\StringTok{"red"}\NormalTok{)}
\KeywordTok{plot}\NormalTok{(returns}\OperatorTok{$}\StringTok{\textasciigrave{}}\DataTypeTok{AB INBEV}\StringTok{\textasciigrave{}}\NormalTok{, }\DataTypeTok{main =} \StringTok{"ABI.BR"}\NormalTok{,}\DataTypeTok{col=}\StringTok{"red"}\NormalTok{)}
\KeywordTok{plot}\NormalTok{(returns}\OperatorTok{$}\NormalTok{DAIMLER, }\DataTypeTok{main =} \StringTok{"DAI.DE"}\NormalTok{,}\DataTypeTok{col=}\StringTok{"red"}\NormalTok{)}
\KeywordTok{plot}\NormalTok{(returns}\OperatorTok{$}\NormalTok{ORANGE, }\DataTypeTok{main =} \StringTok{"ORA.PA"}\NormalTok{,}\DataTypeTok{col=}\StringTok{"red"}\NormalTok{)}
\KeywordTok{plot}\NormalTok{(returns}\OperatorTok{$}\NormalTok{SANOFI, }\DataTypeTok{main =} \StringTok{"SAN.PA"}\NormalTok{,}\DataTypeTok{col=}\StringTok{"red"}\NormalTok{)}
\KeywordTok{plot}\NormalTok{(returns}\OperatorTok{$}\NormalTok{AIRBUS, }\DataTypeTok{main =} \StringTok{"AIR.PA"}\NormalTok{,}\DataTypeTok{col=}\StringTok{"red"}\NormalTok{)}
\KeywordTok{plot}\NormalTok{(returns}\OperatorTok{$}\NormalTok{INTESA, }\DataTypeTok{main =} \StringTok{"ISP.MI"}\NormalTok{,}\DataTypeTok{col=}\StringTok{"red"}\NormalTok{)}
\KeywordTok{plot}\NormalTok{(returns}\OperatorTok{$}\NormalTok{AHOLD, }\DataTypeTok{main =} \StringTok{"AD.AS"}\NormalTok{,}\DataTypeTok{col=}\StringTok{"red"}\NormalTok{)}
\KeywordTok{plot}\NormalTok{(returns}\OperatorTok{$}\NormalTok{ASML, }\DataTypeTok{main =} \StringTok{"ASML.AS"}\NormalTok{,}\DataTypeTok{col=}\StringTok{"red"}\NormalTok{)}
\end{Highlighting}
\end{Shaded}

\includegraphics[width=0.3\linewidth]{bookdown-demo_files/figure-latex/unnamed-chunk-5-1} \includegraphics[width=0.3\linewidth]{bookdown-demo_files/figure-latex/unnamed-chunk-5-2} \includegraphics[width=0.3\linewidth]{bookdown-demo_files/figure-latex/unnamed-chunk-5-3} \includegraphics[width=0.3\linewidth]{bookdown-demo_files/figure-latex/unnamed-chunk-5-4} \includegraphics[width=0.3\linewidth]{bookdown-demo_files/figure-latex/unnamed-chunk-5-5} \includegraphics[width=0.3\linewidth]{bookdown-demo_files/figure-latex/unnamed-chunk-5-6} \includegraphics[width=0.3\linewidth]{bookdown-demo_files/figure-latex/unnamed-chunk-5-7} \includegraphics[width=0.3\linewidth]{bookdown-demo_files/figure-latex/unnamed-chunk-5-8} \includegraphics[width=0.3\linewidth]{bookdown-demo_files/figure-latex/unnamed-chunk-5-9} \includegraphics[width=0.3\linewidth]{bookdown-demo_files/figure-latex/unnamed-chunk-5-10} \includegraphics[width=0.3\linewidth]{bookdown-demo_files/figure-latex/unnamed-chunk-5-11} \includegraphics[width=0.3\linewidth]{bookdown-demo_files/figure-latex/unnamed-chunk-5-12} \includegraphics[width=0.3\linewidth]{bookdown-demo_files/figure-latex/unnamed-chunk-5-13} \includegraphics[width=0.3\linewidth]{bookdown-demo_files/figure-latex/unnamed-chunk-5-14}
\normalsize

\scriptsize

\begin{Shaded}
\begin{Highlighting}[]
 \KeywordTok{par}\NormalTok{(}\DataTypeTok{mar =} \KeywordTok{c}\NormalTok{(}\DecValTok{2}\NormalTok{,}\DecValTok{2}\NormalTok{,}\DecValTok{2}\NormalTok{,}\DecValTok{2}\NormalTok{))}
 \KeywordTok{hist}\NormalTok{(returns}\OperatorTok{$}\NormalTok{TELEFONICA,}\DataTypeTok{probability=}\NormalTok{T, }\DataTypeTok{main=}\StringTok{"TELEFONICA {-} daily.}
\StringTok{      returns"}\NormalTok{,}\DataTypeTok{xlab=}\StringTok{"Approximately normally distributed data"}\NormalTok{,}\DataTypeTok{breaks=}\DecValTok{100}\NormalTok{)}
 \KeywordTok{lines}\NormalTok{(}\KeywordTok{density}\NormalTok{(}\KeywordTok{na.omit}\NormalTok{(returns}\OperatorTok{$}\NormalTok{TELEFONICA)),}\DataTypeTok{col=}\DecValTok{2}\NormalTok{)}
 \KeywordTok{curve}\NormalTok{(}\KeywordTok{dnorm}\NormalTok{(x,}\DecValTok{0}\NormalTok{,}\FloatTok{0.01640906}\NormalTok{), }\DataTypeTok{from =} \FloatTok{{-}0.15}\NormalTok{,}\DataTypeTok{to=}\FloatTok{0.15}\NormalTok{, }\DataTypeTok{col=}\StringTok{\textquotesingle{}blue\textquotesingle{}}\NormalTok{,}\DataTypeTok{add =} \OtherTok{TRUE}\NormalTok{)}
 \KeywordTok{qqnorm}\NormalTok{(returns}\OperatorTok{$}\NormalTok{TELEFONICA,}\DataTypeTok{main=}\StringTok{"QQ plot of normal data"}\NormalTok{,}\DataTypeTok{pch=}\DecValTok{19}\NormalTok{)}
 \KeywordTok{qqline}\NormalTok{(returns}\OperatorTok{$}\NormalTok{TELEFONICA)}
\end{Highlighting}
\end{Shaded}

\includegraphics[width=0.3\linewidth]{bookdown-demo_files/figure-latex/unnamed-chunk-6-1} \includegraphics[width=0.3\linewidth]{bookdown-demo_files/figure-latex/unnamed-chunk-6-2}
\normalsize

\scriptsize

\begin{Shaded}
\begin{Highlighting}[]
\NormalTok{ TELTestNOR \textless{}{-}}\StringTok{ }\KeywordTok{shapiro.test}\NormalTok{(}\KeywordTok{as.numeric}\NormalTok{(returns}\OperatorTok{$}\NormalTok{TELEFONICA))}
\NormalTok{ TELSha \textless{}{-}}\StringTok{ }\KeywordTok{matrix}\NormalTok{(}\KeywordTok{c}\NormalTok{(TELTestNOR}\OperatorTok{$}\NormalTok{statistic,TELTestNOR}\OperatorTok{$}\NormalTok{p.value),}\DataTypeTok{ncol =} \DecValTok{2}\NormalTok{)}
 \KeywordTok{colnames}\NormalTok{(TELSha) \textless{}{-}}\StringTok{ }\KeywordTok{c}\NormalTok{(}\StringTok{"Estatistica"}\NormalTok{,}\StringTok{"p{-}value"}\NormalTok{)}
 \KeywordTok{rownames}\NormalTok{(TELSha) \textless{}{-}}\StringTok{ }\KeywordTok{c}\NormalTok{(}\StringTok{"Shapiro{-}Wilk normality test"}\NormalTok{)}
\NormalTok{ TELSha \textless{}{-}}\StringTok{ }\KeywordTok{as.table}\NormalTok{(TELSha)}
\NormalTok{ d \textless{}{-}}\StringTok{ }\NormalTok{knitr}\OperatorTok{::}\KeywordTok{kable}\NormalTok{(}
\NormalTok{  TELSha, }\DataTypeTok{caption =} \StringTok{\textquotesingle{}Empresas extraidas do Euro Stoxx 50\textquotesingle{}}\NormalTok{,}
  \DataTypeTok{booktabs =} \OtherTok{TRUE}
\NormalTok{)}
\KeywordTok{kable\_styling}\NormalTok{(d, }\DataTypeTok{latex\_options =} \StringTok{"hold\_position"}\NormalTok{, }\DataTypeTok{position =} \StringTok{"center"}\NormalTok{)}
\end{Highlighting}
\end{Shaded}

\begin{table}[!h]

\caption{\label{tab:unnamed-chunk-7}Empresas extraidas do Euro Stoxx 50}
\centering
\begin{tabular}[t]{lrr}
\toprule
  & Estatistica & p-value\\
\midrule
Shapiro-Wilk normality test & 0.889425 & 0\\
\bottomrule
\end{tabular}
\end{table}
\normalsize
\scriptsize

\normalsize

Como se pode verificar, o desvio padrão da empresa telecomunicação ``Telefonica'' é 0.0164091

\hypertarget{conclusuxe3o}{%
\chapter*{Conclusão}\label{conclusuxe3o}}
\addcontentsline{toc}{chapter}{Conclusão}

We have finished a nice book.

  \bibliography{book.bib,packages.bib}

\chapter*{Apêndice}
\addcontentsline{toc}{chapter}{Apêndice}


\end{document}
