% Options for packages loaded elsewhere
\PassOptionsToPackage{unicode}{hyperref}
\PassOptionsToPackage{hyphens}{url}
%
\documentclass[
  12pt,
  brazil,
  a4paper,
  openany]{book}
\usepackage{lmodern}
\usepackage{setspace}
\usepackage{amssymb,amsmath}
\usepackage{ifxetex,ifluatex}
\ifnum 0\ifxetex 1\fi\ifluatex 1\fi=0 % if pdftex
  \usepackage[T1]{fontenc}
  \usepackage[utf8]{inputenc}
  \usepackage{textcomp} % provide euro and other symbols
\else % if luatex or xetex
  \usepackage{unicode-math}
  \defaultfontfeatures{Scale=MatchLowercase}
  \defaultfontfeatures[\rmfamily]{Ligatures=TeX,Scale=1}
  \setmainfont[]{Arial}
\fi
% Use upquote if available, for straight quotes in verbatim environments
\IfFileExists{upquote.sty}{\usepackage{upquote}}{}
\IfFileExists{microtype.sty}{% use microtype if available
  \usepackage[]{microtype}
  \UseMicrotypeSet[protrusion]{basicmath} % disable protrusion for tt fonts
}{}
\usepackage{xcolor}
\IfFileExists{xurl.sty}{\usepackage{xurl}}{} % add URL line breaks if available
\IfFileExists{bookmark.sty}{\usepackage{bookmark}}{\usepackage{hyperref}}
\hypersetup{
  pdflang={pt-br},
  hidelinks,
  pdfcreator={LaTeX via pandoc}}
\urlstyle{same} % disable monospaced font for URLs
\usepackage[left=3.5cm,right=2cm,top=3cm,bottom=3cm, asymmetric]{geometry}
\usepackage{color}
\usepackage{fancyvrb}
\newcommand{\VerbBar}{|}
\newcommand{\VERB}{\Verb[commandchars=\\\{\}]}
\DefineVerbatimEnvironment{Highlighting}{Verbatim}{commandchars=\\\{\}}
% Add ',fontsize=\small' for more characters per line
\usepackage{framed}
\definecolor{shadecolor}{RGB}{248,248,248}
\newenvironment{Shaded}{\begin{snugshade}}{\end{snugshade}}
\newcommand{\AlertTok}[1]{\textcolor[rgb]{0.94,0.16,0.16}{#1}}
\newcommand{\AnnotationTok}[1]{\textcolor[rgb]{0.56,0.35,0.01}{\textbf{\textit{#1}}}}
\newcommand{\AttributeTok}[1]{\textcolor[rgb]{0.77,0.63,0.00}{#1}}
\newcommand{\BaseNTok}[1]{\textcolor[rgb]{0.00,0.00,0.81}{#1}}
\newcommand{\BuiltInTok}[1]{#1}
\newcommand{\CharTok}[1]{\textcolor[rgb]{0.31,0.60,0.02}{#1}}
\newcommand{\CommentTok}[1]{\textcolor[rgb]{0.56,0.35,0.01}{\textit{#1}}}
\newcommand{\CommentVarTok}[1]{\textcolor[rgb]{0.56,0.35,0.01}{\textbf{\textit{#1}}}}
\newcommand{\ConstantTok}[1]{\textcolor[rgb]{0.00,0.00,0.00}{#1}}
\newcommand{\ControlFlowTok}[1]{\textcolor[rgb]{0.13,0.29,0.53}{\textbf{#1}}}
\newcommand{\DataTypeTok}[1]{\textcolor[rgb]{0.13,0.29,0.53}{#1}}
\newcommand{\DecValTok}[1]{\textcolor[rgb]{0.00,0.00,0.81}{#1}}
\newcommand{\DocumentationTok}[1]{\textcolor[rgb]{0.56,0.35,0.01}{\textbf{\textit{#1}}}}
\newcommand{\ErrorTok}[1]{\textcolor[rgb]{0.64,0.00,0.00}{\textbf{#1}}}
\newcommand{\ExtensionTok}[1]{#1}
\newcommand{\FloatTok}[1]{\textcolor[rgb]{0.00,0.00,0.81}{#1}}
\newcommand{\FunctionTok}[1]{\textcolor[rgb]{0.00,0.00,0.00}{#1}}
\newcommand{\ImportTok}[1]{#1}
\newcommand{\InformationTok}[1]{\textcolor[rgb]{0.56,0.35,0.01}{\textbf{\textit{#1}}}}
\newcommand{\KeywordTok}[1]{\textcolor[rgb]{0.13,0.29,0.53}{\textbf{#1}}}
\newcommand{\NormalTok}[1]{#1}
\newcommand{\OperatorTok}[1]{\textcolor[rgb]{0.81,0.36,0.00}{\textbf{#1}}}
\newcommand{\OtherTok}[1]{\textcolor[rgb]{0.56,0.35,0.01}{#1}}
\newcommand{\PreprocessorTok}[1]{\textcolor[rgb]{0.56,0.35,0.01}{\textit{#1}}}
\newcommand{\RegionMarkerTok}[1]{#1}
\newcommand{\SpecialCharTok}[1]{\textcolor[rgb]{0.00,0.00,0.00}{#1}}
\newcommand{\SpecialStringTok}[1]{\textcolor[rgb]{0.31,0.60,0.02}{#1}}
\newcommand{\StringTok}[1]{\textcolor[rgb]{0.31,0.60,0.02}{#1}}
\newcommand{\VariableTok}[1]{\textcolor[rgb]{0.00,0.00,0.00}{#1}}
\newcommand{\VerbatimStringTok}[1]{\textcolor[rgb]{0.31,0.60,0.02}{#1}}
\newcommand{\WarningTok}[1]{\textcolor[rgb]{0.56,0.35,0.01}{\textbf{\textit{#1}}}}
\usepackage{longtable,booktabs}
% Correct order of tables after \paragraph or \subparagraph
\usepackage{etoolbox}
\makeatletter
\patchcmd\longtable{\par}{\if@noskipsec\mbox{}\fi\par}{}{}
\makeatother
% Allow footnotes in longtable head/foot
\IfFileExists{footnotehyper.sty}{\usepackage{footnotehyper}}{\usepackage{footnote}}
\makesavenoteenv{longtable}
\usepackage{graphicx}
\makeatletter
\def\maxwidth{\ifdim\Gin@nat@width>\linewidth\linewidth\else\Gin@nat@width\fi}
\def\maxheight{\ifdim\Gin@nat@height>\textheight\textheight\else\Gin@nat@height\fi}
\makeatother
% Scale images if necessary, so that they will not overflow the page
% margins by default, and it is still possible to overwrite the defaults
% using explicit options in \includegraphics[width, height, ...]{}
\setkeys{Gin}{width=\maxwidth,height=\maxheight,keepaspectratio}
% Set default figure placement to htbp
\makeatletter
\def\fps@figure{htbp}
\makeatother
\setlength{\emergencystretch}{3em} % prevent overfull lines
\providecommand{\tightlist}{%
  \setlength{\itemsep}{0pt}\setlength{\parskip}{0pt}}
\setcounter{secnumdepth}{5}
\pagestyle{plain}
\usepackage{booktabs}
\usepackage{amsmath,amsfonts,amsthm,bm} % Math packages
\usepackage{fontspec}
\usepackage[utf8]{inputenc}
\usepackage[portuguese]{babel}

\usepackage{etoolbox}

\usepackage{titlesec}

\makeatletter
\patchcmd{\ttl@save@mkschap}{*}{}{}{}
\makeatother

\usepackage{indentfirst}
\ifxetex
  % Load polyglossia as late as possible: uses bidi with RTL langages (e.g. Hebrew, Arabic)
  \usepackage{polyglossia}
  \setmainlanguage[]{brazil}
\else
  \usepackage[shorthands=off,main=brazil]{babel}
\fi
\ifluatex
  \usepackage{selnolig}  % disable illegal ligatures
\fi
\usepackage[style=apa,]{biblatex}
\addbibresource{book.bib}
\addbibresource{packages.bib}

\author{}
\date{\vspace{-2.5em}}

\begin{document}

%titlepage
\thispagestyle{empty}
\begin{center}
\begin{minipage}{0.90\linewidth}
    \centering
%University logo
    \includegraphics[width=1\textwidth]{image/biglogo.png}

%Thesis title
    {\uppercase{\Large Impacto da volatilidade na otimização de portfolios financeiros\par}}
    \vspace{2cm}
%Author's name
    {\Large Leonel da Silva Baptista\par}
    \vspace{2cm}
%Degree
    {\Large Mestrado em Estatística, Matemática e Computação\par}
    {\Large Ramo Estatística Computacional\par}
    \vspace{2cm}

%Date
    {\Large 2020}
\end{minipage}
\end{center}


%titlepage
\thispagestyle{empty}
\begin{center}
\begin{minipage}{0.90\linewidth}
    \centering
%University logo
    \includegraphics[width=1\textwidth]{image/logo.png}

%Thesis title
    {\uppercase{\Large Impacto da volatilidade na otimização de portfolios financeiros\par}}
    \vspace{2cm}
%Author's name
    {\Large Leonel da Silva Baptista\par}
    \vspace{2cm}
%Degree
    {\Large Mestrado em Estatística, Matemática e Computação\par}
    {\Large Ramo Estatística Computacional\par}
    \vspace{2cm}
%oriented
    {\Large Dissertação orientada pelo\par}
    {\Large Professor Doutor Amílcar Manuel do Rosário Oliveira\par}
    \vspace{2cm}
%Date
    {\Large 2020}
\end{minipage}
\end{center}

\pagenumbering{gobble}% Remove page numbers (and reset to 1)

\clearpage

\chapter*{Resumo}
\fontsize{12}{21}\selectfont
A presente dissertação têm como âmbito a análise de três métodos diferentes de obter a volatilidade de instrumentos financeiros, nomeadamente valores mobiliários, e seu consequente impacto no resultado da rentabilidade de portfolios constituídos utilizando como pressuposto a média variância, assim como a sua exposição ao risco, sendo que a volatilidade constitui uma peça central na constituição de determinados instrumentos financeiros e respetivo calculo de exposição ao risco.

Os métodos analisados para calculo da volatilidade são a Média Movel Exponencial (EWMA), o modelo da heteroscedasticidade condicional auto-regressiva generalizada (GARCH) e a volatilidade implícita. Os dois primeiros métodos têm como subjacentes dados históricos dos instrumentos financeiros, sendo que a volatilidade implícita é a volatilidade esperada pelo mercado, sendo obtida através da cotação das opções dos respetivos subjacentes.

A análise de risco é efetuada aplicando dois métodos complementares de análise. O Value at Risk (VaR), que contempla a percentagem de perdas que excedem o VaR e o Expected shortfall (ES), que contempla a magnitude dessas perdas.

Este trabalho é realizado tendo como ferramenta de apoio a linguagem de programação R.
\bigbreak

\noindent\textbf{Palavras chave:} Volatilidade, portfolio, rentabilidade, risco, R




\pagenumbering{roman}% Arabic page numbers (and reset to 1)
\setcounter{page}{2}

\chapter*{Abstract}
\fontsize{12}{21}\selectfont
The scope of this dissertation is to analyze three different methods of obtaining the volatility of financial instruments, namely securities, and their consequent impact on the profitability of portfolios constituted using the assumption of mean-variance, as well as their exposure to risk, volatility is a central element in the constitution of certain financial instruments and the respective calculation of exposure to risk.

The methods analyzed for calculating the volatility are the Exponential Moving Average (EWMA), the generalized autoregressive conditional heteroscedasticity model (GARCH) and the implied volatility. The first two methods are based on historical data on financial instruments, the implied volatility being the volatility expected by the market, being obtained through the quotation of the options of the respective underlying.

The risk analysis is carried out using two complementary methods of analysis. Value at Risk (VaR), which includes the percentage of losses that exceed VaR and Expected shortfall (ES), which considers the magnitude of these losses.

This work is carried out using the R programming language as a support tool.

\bigbreak

\noindent\textbf{Keywords:} Volatility, portfolio, profitability, risk, R

\newenvironment{dedication}
  {\clearpage           % we want a new page
   \itshape             % the text is in italics
   \raggedleft          % flush to the right margin
  }
  {\par % end the paragraph
   \vspace{\stretch{3}} % space at bottom is three times that at the top
   \clearpage           % finish off the page
  }
\begin{dedication}
{\Large Dedicado a minha esposa\par}
\end{dedication}

\chapter*{Agradecimentos}

\renewcommand*\contentsname{Índice}
{
\setcounter{tocdepth}{2}
\tableofcontents
}
\listoftables
\listoffigures
\setstretch{1.5}
\hypertarget{simbologia-e-notauxe7uxf5es}{%
\chapter*{Simbologia e notações}\label{simbologia-e-notauxe7uxf5es}}

\mainmatter

\hypertarget{intro}{%
\chapter*{Introdução}\label{intro}}
\addcontentsline{toc}{chapter}{Introdução}

A estatística aplicada ao sector financeiro têm sido pratica comum nas últimas décadas, sendo que a sua aplicação não se pode resumir apenas a estatística descritiva, tendo vindo a beneficiar dos avanços verificados na aplicação de ferramentas estatísticas para analise preditiva dos dados, devido essencialmente aos avanços tecnológicos no hardware de equipamentos informáticos que permitem a aplicação de algoritmos mais complexos que de outro modo não seria possível utilizar, pelo menos em tempo útil.

Um dos sectores com grande aplicabilidade dos métodos matemáticos e estatísticos nas finanças é a analise quantitativa, sendo as principais áreas de aplicação a estruturação de derivados, gestão do risco, trading automático e gestão de investimentos.

Historicamente, a analise quantitativa iniciou-se em 1900 com Louis Jean-Baptiste Alphonse Bachelier, onde a sua tese de doutoramento forneceu um modelo para estipular o preço de opções considerando uma distribuição normal (\url{https://en.wikipedia.org/wiki/Quantitative_analysis_(finance)}).

Na década de 50, Harry Markowitz escreveu um artigo intitulado ``Portfolio Selection'' que viria a revolucionar o modo como selecionar uma carteira de instrumentos financeiros, aplicando princípios de correlação e variância de modo a constituir portfolios de ações , onde a ``fronteira eficiente'' representa portfolios que maximizam retornos de acordo com o risco assumido, providenciando modelos que demonstravam que só com a diversificação de investimentos é que se conseguiria atingir a eficiência, embora só bastante mais tarde esta teoria começa-se a ver a sua aplicabilidade nas instituições financeiras. A aplicabilidade de técnicas estatísticas fica saliente quando Markowitz estipula que ``para usar a regra da média-variância na seleção de ações devemos ter procedimentos para encontrar \(\symbf{\mu_i}\) e \(\symbf{\sigma_{ij}}\) razoáveis. Estes procedimentos, eu acredito, devem combinar técnicas estatísticas e julgamento prático do Homem'' \autocite[pp.91]{Markowitz1952}. As limitações do modelo prendem-se com pressuposto que não representam exatamente a realidade, como o pressuposto de que os retornos das ações seguem uma distribuição normal, sendo que a distribuição dos retornos segue muitas vezes uma curtose leptocúrtica, apresentando caudas pesadas.

Na década de 60, Sharpe, Lintner and Mossin desenvolveram um modelo para equilíbrio de mercado, definido como ``Capital Asset Pricing Model''(CAPM), descrevendo a relação entre risco sistemático e o retorno esperado. O modelo pressupõe que, se todos os investidores contêm o mesmo portfolio, então, em equilíbrio esse deve ser o portfolio de mercado. De acordo com \textcite{Sharpe1964} em equilíbrio os preços dos ativos de capital foram ajustados de forma que o investidor consiga atingir qualquer ponto desejado ao longo da reta do mercado capital, ou ``capital market line'' (CMP), pressupondo que o investidor siga uma estratégia de diversificação do investimento. O CMP pode ser utilizado de modo a otimizar um portfolio, sendo o ponto tangente a curva denominada de fronteira eficiente.

Também na década de 60 foi apresentado pela primeira vez o modelo de Black-Scholes-Merton, fornecendo uma solução para valorizar opções europeias e outros derivados. O modelo assume que os preços têm uma distribuição lognormal e que a volatilidade é constante ao longo do tempo. A volatilidade que é assumida neste modelo é a volatilidade implícita da opção, ou seja, a volatilidade para o qual o valor dado pelo modelo Black-Scholes-Merton iguala o preço de mercado. Outros modelos foram desenvolvidos de modo a valorizar opções e outros derivados financeiros, entre eles a arvore binomial e simulação de Monte Carlo.

\begingroup
\titleformat{\chapter}[display]
{\normalfont\huge\bfseries\centering}{\chaptertitlename\ \thechapter}{20pt}{\Huge}

\hypertarget{modelauxe7uxe3o-estatuxedstica-na-otimizauxe7uxe3o-de-portfuxf3lios}{%
\chapter{Modelação Estatística na Otimização de Portfólios}\label{modelauxe7uxe3o-estatuxedstica-na-otimizauxe7uxe3o-de-portfuxf3lios}}

\newpage

You can label chapter and section titles using \texttt{\{\#label\}} after them, e.g., we can reference Chapter \ref{intro}. If you do not manually label them, there will be automatic labels anyway, e.g., Chapter \ref{methods}.

Figures and tables with captions will be placed in \texttt{figure} and \texttt{table} environments, respectively

\begin{Shaded}
\begin{Highlighting}[]
\KeywordTok{par}\NormalTok{(}\DataTypeTok{mar =} \KeywordTok{c}\NormalTok{(}\DecValTok{4}\NormalTok{, }\DecValTok{4}\NormalTok{, }\FloatTok{.1}\NormalTok{, }\FloatTok{.1}\NormalTok{))}
\KeywordTok{plot}\NormalTok{(pressure, }\DataTypeTok{type =} \StringTok{\textquotesingle{}b\textquotesingle{}}\NormalTok{, }\DataTypeTok{pch =} \DecValTok{19}\NormalTok{)}
\end{Highlighting}
\end{Shaded}

\begin{figure}

{\centering \includegraphics[width=0.8\linewidth]{bookdown-demo_files/figure-latex/nice-fig-1} 

}

\caption{Here is a nice figure!}\label{fig:nice-fig}
\end{figure}

Reference a figure by its code chunk label with the \texttt{fig:} prefix, e.g., see Figure \ref{fig:nice-fig}. Similarly, you can reference tables generated from \texttt{knitr::kable()}, e.g., see Table \ref{tab:nice-tab}.

\begin{Shaded}
\begin{Highlighting}[]
\NormalTok{knitr}\OperatorTok{::}\KeywordTok{kable}\NormalTok{(}
  \KeywordTok{head}\NormalTok{(iris, }\DecValTok{20}\NormalTok{), }\DataTypeTok{caption =} \StringTok{\textquotesingle{}Here is a nice table!\textquotesingle{}}\NormalTok{,}
  \DataTypeTok{booktabs =} \OtherTok{TRUE}
\NormalTok{)}
\end{Highlighting}
\end{Shaded}

\begin{table}

\caption{\label{tab:nice-tab}Here is a nice table!}
\centering
\begin{tabular}[t]{rrrrl}
\toprule
Sepal.Length & Sepal.Width & Petal.Length & Petal.Width & Species\\
\midrule
5.1 & 3.5 & 1.4 & 0.2 & setosa\\
4.9 & 3.0 & 1.4 & 0.2 & setosa\\
4.7 & 3.2 & 1.3 & 0.2 & setosa\\
4.6 & 3.1 & 1.5 & 0.2 & setosa\\
5.0 & 3.6 & 1.4 & 0.2 & setosa\\
\addlinespace
5.4 & 3.9 & 1.7 & 0.4 & setosa\\
4.6 & 3.4 & 1.4 & 0.3 & setosa\\
5.0 & 3.4 & 1.5 & 0.2 & setosa\\
4.4 & 2.9 & 1.4 & 0.2 & setosa\\
4.9 & 3.1 & 1.5 & 0.1 & setosa\\
\addlinespace
5.4 & 3.7 & 1.5 & 0.2 & setosa\\
4.8 & 3.4 & 1.6 & 0.2 & setosa\\
4.8 & 3.0 & 1.4 & 0.1 & setosa\\
4.3 & 3.0 & 1.1 & 0.1 & setosa\\
5.8 & 4.0 & 1.2 & 0.2 & setosa\\
\addlinespace
5.7 & 4.4 & 1.5 & 0.4 & setosa\\
5.4 & 3.9 & 1.3 & 0.4 & setosa\\
5.1 & 3.5 & 1.4 & 0.3 & setosa\\
5.7 & 3.8 & 1.7 & 0.3 & setosa\\
5.1 & 3.8 & 1.5 & 0.3 & setosa\\
\bottomrule
\end{tabular}
\end{table}

You can write citations,\autocite[pp.33-85]{Volatility2001}

\hypertarget{pacotes-do-r-para-anuxe1lise}{%
\chapter{Pacotes do R para análise}\label{pacotes-do-r-para-anuxe1lise}}

\newpage

We describe our methods in this chapter.

\hypertarget{aplicauxe7uxe3o-a-dados-do-modelo}{%
\chapter{Aplicação a dados do modelo}\label{aplicauxe7uxe3o-a-dados-do-modelo}}

\endgroup
\newpage

Some \emph{significant} applications are demonstrated in this chapter.

\hypertarget{example-one}{%
\section{Example one}\label{example-one}}

\hypertarget{example-two}{%
\section{Example two}\label{example-two}}

\hypertarget{conclusuxe3o}{%
\chapter*{Conclusão}\label{conclusuxe3o}}
\addcontentsline{toc}{chapter}{Conclusão}

We have finished a nice book.

\printbibliography

\end{document}
