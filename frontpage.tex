%titlepage
\thispagestyle{empty}
\begin{center}
\begin{minipage}{0.90\linewidth}
    \centering
%University logo
    \includegraphics[width=1\textwidth]{image/biglogo.png}

%Thesis title
    {\uppercase{\Large Impacto da volatilidade na otimização de portfolios financeiros\par}}
    \vspace{2cm}
%Author's name
    {\Large Leonel da Silva Baptista\par}
    \vspace{2cm}
%Degree
    {\Large Mestrado em Estatística, Matemática e Computação\par}
    {\Large Ramo Estatística Computacional\par}
    \vspace{2cm}

%Date
    {\Large 2020}
\end{minipage}
\end{center}

\pagenumbering{gobble}% Remove page numbers (and reset to 1)
%titlepage
\thispagestyle{empty}
\begin{center}
\begin{minipage}{0.90\linewidth}
    \centering
%University logo
    \includegraphics[width=1\textwidth]{image/logo.png}

%Thesis title
    {\uppercase{\Large Impacto da volatilidade na otimização de portfolios financeiros\par}}
    \vspace{2cm}
%Author's name
    {\Large Leonel da Silva Baptista\par}
    \vspace{2cm}
%Degree
    {\Large Mestrado em Estatística, Matemática e Computação\par}
    {\Large Ramo Estatística Computacional\par}
    \vspace{2cm}
%oriented
    {\Large Dissertação orientada pelo\par}
    {\Large Professor Doutor Amílcar Manuel do Rosário Oliveira\par}
    \vspace{2cm}
%Date
    {\Large 2020}
\end{minipage}
\end{center}

\pagenumbering{roman}% Arabic page numbers (and reset to 1)
\setcounter{page}{2}

\clearpage

\chapter*{Resumo}
\fontsize{12}{21}\selectfont
O problema a ser tratado nesta dissertação será o impacto dos vários métodos de cálculo da volatilidade no cálculo da rentabilidade ótima do portfolio para um determinado nível de risco.
Desta forma, ao definir o portfolio de ativos financeiros, será analisado o método da média variância na construção dos portfolios, assim como o impacto do número de ativos no valor da volatilidade final do portfolio. O cálculo da variância dos ativos será calculado com base nos dados históricos, na volatilidade implícita, no modelo GARCH e ARCH noutros métodos que possam ser de interesse ao longo do estudo teórico.

historical volatility, EWMA and GARCH.
Será realizada uma análise do VaR (Value at Risk) e do ES (Expected Shortfal) para cada um dos ativos e portfolios obtidos, utilizando diferentes métodos para cálculo do VaR e ES, classificando-se estes em métodos paramétricos e não paramétricos. Desta forma, a análise da otimização dos portfolios será enquadrada com a gestão de risco, VaR e ES, sendo que esta métrica combina duas características, a suposição do tipo de distribuição dos retornos e a previsão da variância.
A análise dos valores obtidos nos portfolios será realizado comparando a rentabilidade obtida nos diferentes portfolios com a rentabilidade obtida no benchmark – ou índice de referência - sendo este um ponto de referência para identificar e comparar o desempenho dos investimentos realizados.
Este trabalho será realizado tendo como ferramenta de apoio a linguagem de programação R.

Este trabalho demonstra que os modelos da família GARCH são mais apropriados para se modelar a variância condicional (volatilidade) de séries de retornos financeiros que apresentam variância condicional evoluindo no tempo em comparação com os modelos ARIMA tradicionais.  ARIMA-GARCH



\chapter*{Abstract}

\newenvironment{dedication}
  {\clearpage           % we want a new page
   \itshape             % the text is in italics
   \raggedleft          % flush to the right margin
  }
  {\par % end the paragraph
   \vspace{\stretch{3}} % space at bottom is three times that at the top
   \clearpage           % finish off the page
  }
\begin{dedication}
{\Large Dedicado a minha esposa\par}
\end{dedication}

\chapter*{Agradecimentos}
