%titlepage
\thispagestyle{empty}
\begin{center}
\begin{minipage}{0.90\linewidth}
    \centering
%University logo
    \includegraphics[width=1\textwidth]{image/biglogo.png}

%Thesis title
    {\uppercase{\Large Impacto da volatilidade na otimização de portfolios financeiros\par}}
    \vspace{2cm}
%Author's name
    {\Large Leonel da Silva Baptista\par}
    \vspace{2cm}
%Degree
    {\Large Mestrado em Estatística, Matemática e Computação\par}
    {\Large Ramo Estatística Computacional\par}
    \vspace{2cm}

%Date
    {\Large 2020}
\end{minipage}
\end{center}


%titlepage
\thispagestyle{empty}
\begin{center}
\begin{minipage}{0.90\linewidth}
    \centering
%University logo
    \includegraphics[width=1\textwidth]{image/logo.png}

%Thesis title
    {\uppercase{\Large Impacto da volatilidade na otimização de portfolios financeiros\par}}
    \vspace{2cm}
%Author's name
    {\Large Leonel da Silva Baptista\par}
    \vspace{2cm}
%Degree
    {\Large Mestrado em Estatística, Matemática e Computação\par}
    {\Large Ramo Estatística Computacional\par}
    \vspace{2cm}
%oriented
    {\Large Dissertação orientada pelo\par}
    {\Large Professor Doutor Amílcar Manuel do Rosário Oliveira\par}
    \vspace{2cm}
%Date
    {\Large 2020}
\end{minipage}
\end{center}

\pagenumbering{gobble}% Remove page numbers (and reset to 1)

\clearpage

\chapter*{Resumo}
\fontsize{12}{21}\selectfont
A presente dissertação tem como âmbito a análise de três métodos diferentes de obter a volatilidade de instrumentos financeiros, nomeadamente valores mobiliários, e seu consequente impacto no resultado da rentabilidade de portfolios constituídos utilizando como pressuposto o modelo da média-variância, assim como a sua exposição ao risco, sendo que a volatilidade constitui uma peça central na constituição de determinados instrumentos financeiros e respetivo cálculo de exposição ao risco.

Os métodos analisados para calculo da volatilidade são a média movel exponencial ponderada (exponentially weighted moving average ou \textbf{EWMA}, na sigle em inglês), o modelo da heteroscedasticidade condicional auto-regressiva generalizada (generalized autoregressive conditional heteroskedasticity ou \textbf{GARCH}, na sigle em inglês) e a volatilidade implícita. Os dois primeiros métodos têm como subjacentes dados históricos dos instrumentos financeiros, sendo que a volatilidade implícita é a volatilidade esperada pelo mercado, sendo obtida através da cotação das opções dos respetivos subjacentes.

A análise de risco é efetuada aplicando o método \emph{Value at Risk} (\textbf{VaR}), que contempla a percentagem de perdas que excedem o VaR.

Este trabalho é realizado tendo como ferramenta de apoio a linguagem de programação R, sendo que se pretende também analisar a adequabilidade e disponibilização de pacotes no R para análise financeira.
\bigbreak

\noindent\textbf{Palavras chave:} Séries temporais, Volatilidade, portfolio, rentabilidade, risco, R



\pagenumbering{roman}% Arabic page numbers (and reset to 1)
\setcounter{page}{2}

\chapter*{Abstract}
\fontsize{12}{21}\selectfont
The scope of this dissertation is to analyze three different methods of obtaining the volatility of financial instruments, namely securities, and their consequent impact on the profitability of portfolios constituted using the assumption of mean-variance, as well as their exposure to risk, volatility is a central element in the constitution of certain financial instruments and the respective calculation of exposure to risk.

The methods analyzed for calculating the volatility are the Exponential Moving Average \textbf{EWMA}, the generalized autoregressive conditional heteroscedasticity model \textbf{GARCH} and the implied volatility. The first two methods are based on historical data on financial instruments, the implied volatility being the volatility expected by the market, being obtained through the quotation of the options of the respective underlying.

The risk analysis is carried out using one methods of analysis, the \emph{Value at Risk} (\textbf{VaR}), which includes the percentage of losses that exceed VaR.

This work is carried out using the R programming language as a support tool, being also analyzed the suitability and availability of packages in R for financial analysis.

\bigbreak

\noindent\textbf{Keywords:} Time series, Volatility, portfolio, profitability, risk, R

\newenvironment{dedication}
  {\clearpage           % we want a new page
   \itshape             % the text is in italics
   \raggedleft          % flush to the right margin
  }
  {\par % end the paragraph
   \vspace{\stretch{3}} % space at bottom is three times that at the top
   \clearpage           % finish off the page
  }
\begin{dedication}
{\Large Dedicado a minha esposa\par}
\end{dedication}

\chapter*{Agradecimentos}
